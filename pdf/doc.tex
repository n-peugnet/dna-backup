\documentclass[a4paper]{report}

% Set page dimentions
\usepackage[margin=35mm]{geometry}

% Bibliography
\usepackage[numbers]{natbib}    % more bibliography options
\usepackage[nottoc]{tocbibind}  % add link to table of contents
\bibliographystyle{plainnat}    % more detailled plain bibliography

% Packages for french documents
\usepackage[french]{babel}  % latex rules for french words
\usepackage[utf8]{inputenc} % UTF-8 encoding for special chars
\usepackage[T1]{fontenc}    % T1 font for smooth render of special chars
\DeclareUnicodeCharacter{202F}{\thinspace}

% Define some colors
\usepackage{color}
\definecolor{string}{RGB}{100, 200, 0}
\definecolor{comment}{RGB}{150, 150, 150}
\definecolor{identifier}{RGB}{100, 100, 200}

% Source code style
\usepackage{listings}
\lstset{
	basicstyle=\footnotesize\ttfamily, % sets font style for the code
	frame=single,                 % adds a frame around the code
	showstringspaces=false,       % underline spaces within strings
	tabsize=4,                    % sets default tabsize to 2 spaces
	breaklines=true,              % sets automatic line breaking
	breakatwhitespace=true,       % sets if automatic breaks should only happen at whitespace
	keywordstyle=\color{magenta}, % sets color for keywords
	stringstyle=\color{string},   % sets color for strings
	commentstyle=\color{comment}, % sets color for comments
	emphstyle=\color{identifier}, % sets color for comments
}

% Graphics
\usepackage{graphicx}        % images and figures
\usepackage{subcaption}      % subcaption and subtables
\graphicspath{ {assets} }    % path containing images
\usepackage{tikz}            % to generate graphics
\usetikzlibrary{arrows.meta} % setup arrows
\usetikzlibrary{chains,decorations.pathreplacing} % tiks chains
\tikzstyle{Arrow}=[-{Stealth[scale=1.5]}]

% Complex tables
\usepackage{multirow}
\usepackage{tabularx}
% Custom column types
\usepackage{array}
\newcolumntype{L}{>{\raggedright\arraybackslash}X} % Left-aligned auto-span columns
\newcolumntype{R}{>{\raggedleft\arraybackslash}X} % Right-aligned auto-span columns

% Localised number print (thousands, decimals, etc...)
\usepackage{numprint}

% Directory tree
\usepackage{dirtree}

% Acronyms
\usepackage[printonlyused,footnote]{acronym}

% Hyperlinks
\usepackage{xurl}                % allow word break for url wrapping
\usepackage[
  hidelinks,                     % invisible links
  pagebackref=true               % linkback to citations from bibliography
]{hyperref}                      % create clickable links
\usepackage{doi}                 % add links to doi urls

% Backrefs in the bibliography for french
\renewcommand{\backrefalt}[4]{
\ifcase #1 {}
  \or \emph{Cité page #2}
  \else \emph{Cité pages #2}
\fi
}
% Backref separators for french
\renewcommand{\backreftwosep}{ et~}
\renewcommand{\backreflastsep}{, et~}

% Style
\setlength{\parskip}{.3em} % space between paragraphs

% Custom commands
\usepackage{xspace}
\newcommand{\etal}{\emph{et al.}\@\xspace}

\newcommand{\btrfs}{BTRFS~\cite{rodeh2013btrfs}\@\xspace}
\newcommand{\erofs}{EROFS~\cite{gao2019erofs}\@\xspace}
\newcommand{\hptfs}{HPTFS~\cite{zhang2006hptfs}\@\xspace}
\newcommand{\ltfs}{LTFS~\cite{pease2010linear}\@\xspace}
\newcommand{\squashfs}{SquashFS~\cite{lougher2009squashfs}\@\xspace}
\newcommand{\udf}{UDF~\cite{optical2003universal}\@\xspace}

%--------------------------------------- Content ---------------------------------------%

\title{Système de fichiers pour le stockage d’informations numériques sur ADN}

\date{Octobre 2021}

\author{Nicolas Peugnet}

\begin{document}

\pagenumbering{Roman}
\begin{titlepage}
\maketitle
\end{titlepage}

\pagenumbering{arabic}
\tableofcontents

\chapter{Introduction}

Ce stage a été réalisé dans le cadre du projet DNA-Drive, un système développé par l'équipe de Stéphane Lemaire (\ac{lcqb}) visant à stocker des données numériques arbitraires via des molécules d'\ac{adn}.
Notre rôle sera de proposer un \emph{système de fichiers} adapté aux spécificités de ce nouveau médium de stockage.

\section{Systèmes de fichiers}

Le but principal d'un \emph{système de fichiers} est de permettre d'organiser des données sur un support de stockage.
En son absence, les différentes données d'un support seraient toutes écrites d'un seul tenant
et il ne serait pas possible de savoir où commence et où s'arrête un morceau cohérent de données.
Le système de fichiers offre donc la possibilité de répartir les données dans des segments nommés : les \emph{fichiers}.
Les fichiers peuvent être disposés au sein d'une arborescence de \emph{dossiers},
ce qui permet de les organiser de manière logique.
Les informations de cette arborescence sont des métadonnées que l'on doit donc stocker en plus des données.
Les systèmes de fichiers récents stockent en général bien plus de métadonnées que simplement les noms des fichiers et des dossiers.
Parmi celles-ci, on peut citer par exemple les dates de création et de modification,
ou bien encore les permissions lorsqu'il s'agit d'un système POSIX.

Grâce à cette structure d'arborescence de fichiers commune à tous les systèmes,
on obtient une compatibilité de l'un à l'autre,
ce qui offre à la couche supérieure une abstraction des supports de stockages.

Pour stocker ou lire des données, un système de fichiers doit communiquer avec le matériel.
Lorsqu'il s'agit d'un support de stockage classique respectant le \ac{lba},
comme c'est le cas des disque-durs et des \ac{ssd},
le système de fichiers n'a pas besoin de connaitre le fonctionnement détaillé du périphérique.
Il lui suffit de demander de lire ou écrire les données du bloc $n$
et c'est le contrôleur du périphérique lui-même qui se charge de retrouver l'emplacement physique de ce bloc logique.

Pour des supports plus spécifiques, il est assez fréquent que les systèmes de fichiers soient conçus exclusivement pour eux.
C'est par exemple le cas pour les bandes magnétiques avec \hptfs et \ltfs ou les disques optiques avec \udf.
Ces systèmes seront parfaitement adaptés à leur support et pourront donc pousser plus loin certaines optimisations.

\subsection{Fonctionnalités et caractéristiques}

En plus de son rôle d'abstraction des supports de stockage,
un système de fichier peut proposer un grand nombre de fonctionnalités et de caractéristiques intéressantes.
Nous n'introduirons ici que celles qui nous seront utiles pour la suite.

\paragraph{Type d'accès}
Un système de fichiers peut être accessible soit en lecture uniquement (\ac{ro}),
comme c'est le cas d'une bonne partie des systèmes compressés (\squashfs, \erofs, etc~\textellipsis),
soit en lecture et écriture (\ac{rw}) pour la grande majorité des autres systèmes.

\paragraph{Compression}
La compression peut avoir plusieurs intérêts pour un système de fichiers.
Elle peut bien-sûr permettre de réduire l'espace pris sur le support
et c'est ce cas d'usage que \squashfs tente d'optimiser.
Mais elle peut aussi servir à accélérer les opérations de lecture et d'écriture lorsqu'on est limité par la bande passante.
Les systèmes plus généralistes, comme \btrfs, utilisent la compression surtout dans ce but
et d'autres systèmes de fichiers, comme \erofs, sont même entièrement basés sur ce principe.

% TODO: quelles autres propriétés ?

\section{Stocker des données sur ADN}

L’\ac{adn} ou Acide DésoxyriboNucléique d’un organisme, constitue ce qu’on appelle le génome.
Le génome contient l’information génétique d’un organisme. L’\ac{adn} contient donc une information.
Cette information est codée sous la forme d’une suite de \emph{nucléotides}.
Un nucléotide est une molécule organique qui est l’élément de base de l’\ac{adn}.
Il existe quatre nucléotides différents qui sont représentés par quatre lettres : \textbf{A} pour Adénine, \textbf{C} pour Cytosine, \textbf{G} pour Guanine et \textbf{T} pour Thymine.
Nous pouvons voir directement le parallèle que nous pouvons faire entre l’\ac{adn} qui est une suite de nucléotides en base~4 et une donnée informatique qui est une suite de bits en base~2.

Il est donc naturel de penser à utiliser l’\ac{adn} pour stocker des données
et un certain nombre de démonstrations de faisabilité du stockage sur l’\ac{adn} ont été réalisées lors des dernières années.
Les travaux publiés pour l’instant se basent essentiellement sur l’utilisation d’\emph{oligonucléotides} qui sont des courts segments d’\ac{adn}.

\subsection{Encodages}

Les premières démonstrations significatives sur l’utilisation de ces oligonucléotides pour stocker des données remontent à seulement 2012 avec George Church~\cite{church2012next} qui réussit à stocker 658~ko sur \numprint{54898} oligonucléotides.
Dans ses travaux, Church souhaite pouvoir contrôler le taux de GC et limiter les répétitions d’un même nucléotide.
Le taux de GC est la proportion de nucléotides G et C dans une séquence donnée. 
Les appariements GC ont trois liaisons hydrogène tandis que les appariements AT n'en ont que deux.
Un taux de GC élevé assure ainsi une meilleure stabilité, mais un taux trop élevé peut provoquer une autolyse (autodestruction) plus facilement.
Il est donc préférable d’avoir un taux de GC équilibré.
En ce qui concerne les longues répétitions d’un même nucléotide, elles produisent des erreurs lors du séquençage.
Pour toutes ces raisons, Church va utiliser un encodage en base~2 : $A=C=0$ et $T=G=1$ pour avoir plus de flexibilité (Table~\ref{tab:church-encoding}).

\begin{table}[ht]
\centering
\setlength{\tabcolsep}{.8em}
\renewcommand\arraystretch{1.5}

\begin{subtable}[t]{.45\textwidth}
  \centering
  \begin{tabular}{|c|c|c|c|}
  \hline
  \multirow{2}{*}{\textbf{Bit}} & \textbf{0} & A & C \\
  \cline{2-4}
  & \textbf{1} & T & G \\
  \hline
  \end{tabular}
  \caption{Encodage Church~\cite{church2012next}}
  \label{tab:church-encoding}
\end{subtable}
\hfill
\begin{subtable}[t]{.54\textwidth}
  \centering
  \begin{tabular}{|c|c|c|c|}
  \cline{3-4}
  \multicolumn{2}{c|}{} & \multicolumn{2}{c|}{\textbf{Position}} \\
  \cline{3-4}
  \multicolumn{2}{c|}{} & \textbf{Paire} & \textbf{Impaire} \\
  \hline
  \multirow{2}{*}{\textbf{Bit}} & \textbf{0} & A & C \\
  \cline{2-4}
  & \textbf{1} & T & G \\
  \hline
  \end{tabular}
  \caption{Encodage BIODATA}
  \label{tab:biodata-encoding}
\end{subtable}

\caption{Différents encodages pour \ac{adn}}
\label{tab:dna-encodings}
\end{table}

Suite à ces travaux, un certain nombre de nouvelles publications vont apporter des améliorations intéressantes aux techniques existantes,
avec par exemple l'encodage de Goldman~\cite{goldman2013towards} qui propose l'utilisation d'une base~3 (Figure~\ref{fig:goldman-encoding}),
plus performante en densité de stockage.

\begin{figure}[ht]
\centering
\includegraphics[width=.6\textwidth]{goldman-encoding}
\caption{Encodage de Goldman~\cite{goldman2013towards}}
\label{fig:goldman-encoding}
\end{figure}


\subsection{Spécificités du DNA-Drive}

\subsubsection{Spécificités biologiques}

La spécificité principale de DNA-Drive par rapport à ses concurrents est d'utiliser la molécule d'\ac{adn} sous sa forme de double hélice, plutôt que sous la forme d'un simple brin.

Cette forme a plusieurs avantages.
Premièrement, la molécule est plus stable sous cette forme, ce qui limite sa dégradation et permet donc d'augmenter sa durée de vie.
Deuxièmement, il s'agit de la forme utilisée par l'ensemble des organismes vivants de notre planète\footnote{En considérant que les virus ne sont pas vivants},
ce qui nous permet donc potentiellement de profiter des mécanismes du vivant,
tels que la réparation automatique de l’\ac{adn} pour corriger les erreurs
ou la division cellulaire qui va permettre une copie peu coûteuse et très rapide de grandes quantités de données.

Cependant, faire en sorte qu'une molécule d'\ac{adn} soit compatible avec un être vivant lui ajoute des contraintes supplémentaires.
En particulier, en plus de garantir un taux de GC équilibré,
notre encodeur doit à tout prix éviter que les séquences de données, une fois encodées en ADN,
ne soient interprétés par l'hôte comme des séquences codantes de son génome.

Dans l'ADN d'un être vivant, ces parties codantes sont délimités par deux courtes séquences de nucléotides placées au début et à la fin de la zone à interpréter.
Il s'agit des codons START et STOP.
Le codon START indique le début d'une séquence à interpréter.
C'est donc celui qu'il faut à tout prix éviter d'obtenir une fois les données encodées.
Le codon STOP, au contraire, définit la fin d'une telle séquence.
Il est donc intéressant d'en insérer un maximum pour limiter la casse dans l'éventualité où un codon START aurait malencontreusement été ajouté.

En ce qui concerne la lecture des données, on utilise un séquenceur génétique portatif à
nanopore tels que celui utilisé par l’équipe de H. Yadzi~\cite{yazdi2017portable} et présenté sur la Figure~\ref{fig:oxford-nanopore-minion}.
Les séquenceurs en général ont un problème avec la lecture des homopolymères, c’est-à-dire des
séquences de répétitions d’un même nucléotide. On interdit donc les séquences de plus de trois fois
le même nucléotide pour éviter les erreurs de séquençage.

\begin{figure}[ht]
\centering
\includegraphics[width=.6\textwidth]{oxford-nanopore-minion}
\caption{Lecteur Oxford Nanopore MinION}
\label{fig:oxford-nanopore-minion}
\end{figure}

BIODATA, l'encodage mis au point par Clémence Blachon (\ac{lcqb}) pour le DNA-Drive est justement chargé de faire respecter ces propriétés par les données encodées.
Il est inspiré de celui de Church auquel des contraintes supplémentaires viennent s'appliquer :
Pour chaque bit, on fixe la valeur du nucléotide en fonction de sa valeur et de la parité de sa position (Table~\ref{tab:biodata-encoding}).
De cette manière l'encodage est totalement contraint et le résultat est déterministe.
Les valeurs choisies garantissent quelle que soit la séquence de bits que :

\begin{itemize}
  \item Le taux de GC restera équilibré.
  \item Un nucléotide ne sera jamais répété plusieurs fois à la suite.
  \item Aucun codon START ne sera inséré.
  \item Des codons STOP seront insérés régulièrement.
\end{itemize}

Cet encodage nous permet donc de nous abstraire totalement des problématiques biologiques sous-jacentes lorsqu'on encode des données
et nous laisse ainsi la possibilité de stocker des valeurs complètement arbitraires.

\subsubsection{Spécificités techniques}

L'organisation physique des données du DNA-Drive est assez particulière et doit être prise en compte afin d'optimiser les lectures et les écritures.

\paragraph{Track} Une \emph{track} est un segment de données, actuellement de 1024~octets.
C'est la plus petite unité de stockage du système.
Toutes les écritures devront donc être alignées sur la taille d'une track.
Elles sont obtenues grâce à un assemblage MoClo~\cite{werner2012fast} sur 3 niveaux
et sont refermées en un cercle pour former des \emph{plasmides}.
Cette forme particulière de la molécule d'\ac{adn} comporte plusieurs avantages.
Elle est tout d'abord plus pérenne grâce à sa structure circulaire,
car c'est principalement par ses extrémités que la molécule se dégrade.
Mais les plasmides profitent en plus de leur propre mécanisme d'auto-réplication autonome.
Cependant, cette forme ne permet pas de les mettre bout-à-bout,
il n'y a donc pas d'ordre naturel entre les tracks.
C'est pour cette raison que chaque track contient un \emph{barcode} qui permet de les identifier et donc de les réordonner.

\paragraph{Barcode} Le \emph{barcode} est un entier de 4~octets présent au tout début d'une track et permettant de l'identifier.

\paragraph{Pool} Un \emph{pool} est une minicapsule contenant plusieurs tracks.
Actuellement, un pool peut contenir \numprint{10000} tracks et les lectures sont réalisées par pool entier.
Il est possible de lire plusieurs pools en même temps
et même de fusionner des pools si les barcodes des tracks qu'ils contiennent ne se chevauchent pas.

\paragraph{Array} Un \emph{array} est une plaque de 96~pools ($8\times12$) qui est traditionnellement utilisée en biologie.
La taille maximum disponible pour des données d’un array est donc de \numprint{979.2}~Mo ($96\times\numprint{10000}\times\numprint{1020}$~octets).
On peut multiplier les arrays afin d'obtenir une plus grande capacité de stockage.

Les chiffres donnés ici sur l’organisation du disque de \numprint{1024} octets par track et \numprint{10000} tracks par pool sont limités respectivement par la complexité de l'assemblage MoClo et la précision limitée des techniques de séquençage actuelles.
La capacité disponible est donc amenée à évoluer dans le futur avec l'arrivée de technologies plus performantes.

\section{Problématique}

Il existe donc plusieurs techniques et encodages permettant de stocker des informations arbitraires sur la molécule d'\ac{adn},
mais toutes ont en commun quelques inconvénients majeurs.
Ces inconvénients proviennent pour la plupart des limites actuelles des technologies de synthèse et de séquençage.
Une expérience d'un système automatisé de transmission de données par \ac{adn} datant de 2019~\cite{takahashi2019demonstration}
nous donne un ordre de grandeur des durées de lecture et d'écriture,
bien qu'il ne serait pas étonnant que des techniques plus performantes fassent leur apparition dans un futur proche :
La latence d'une opération d'écriture suivie d'une lecture d'une séquence de 12~octets est de 21~h,
dont \numprint{20.4}~h pour l'écriture (\numprint{8.4}~h de synthèse à 305~s par base et 12~h de stabilisation)
et les 36~min restantes pour la lecture (30~min de préparation et 6~min de séquençage et décodage).
Les lectures ne sont donc déjà pas très rapides, mais le point le plus limitant provient très largement des écritures qui sont exceptionnellement lentes, sans même parler de leur prix.

Une autre inconvénient du DNA-Drive et de l'ensemble des initiatives de stockage de données sur \ac{adn} est l'impossibilité de supprimer ou de modifier des données une fois écrites.
Ce point est particulièrement bloquant pour un système de fichiers \ac{rw} classique,
qui se base sur ces deux propriétés pour mettre à jour les fichiers et leurs métadonnées ainsi que pour récupérer de l'espace lorsque des fichiers sont supprimés.

Cette problématique se retrouve sur d'autres systèmes de stockages, comme les bandes magnétiques ou les disques optiques.
Elle est résolue par leur système de fichiers respectif, \ltfs pour les bandes magnétiques et \udf pour les CD et DVD non-RW.
Dans les deux cas le système est basé sur la réécriture complète des blocs modifiés des fichiers ainsi que de l'index dans le cas de \ltfs ou de la Virtual Allocation Table dans le cas de \udf.

La difficulté principale était donc de réussir à implémenter cette fonctionnalité sur un médium de stockage qui n'a pas la capacité de modifier les données existantes, tout en limitant les écritures au strict nécessaire.


\section{Réponse}

La proposition qui suit s'inscrit dans le cadre d'une réponse à court terme au problème posé.
Nous avons choisi de ne pas nous projeter trop loin dans le temps et avons donc basé l'ensemble de la réflexion sur les capacités actuelles des technologies de synthèse et de séquençage \ac{adn}.

L'objectif principal du système d'archivage de fichiers proposé est de réduire la quantité de données écrites, tout minimisant la quantité de données à lire pour récupérer les données.

Toutes les contraintes citées précédemment nous ont incité % TODO: j'aime bof ce mot
à nous orienter plus vers un système de sauvegardes que vers un véritable système de fichiers.
En effet, les vitesses et coûts d'écriture et de lecture ne permettent, pour le moment,
absolument pas d'en faire un système de fichiers accessible à chaud.
Les cas d'usage envisagés seront donc ceux de sauvegardes sur différentes plages de temps :
journalières, hebdomadaires ou mensuelles.
De cette manière, l'ensemble des opérations réalisés sur les fichiers pendant cette plage de temps
seront factorisées dans un seul bloc de modification : la nouvelle version.
Ce n'est effectivement pas la peine d'écrire un fichier s'il va être supprimé ou renommé quelques secondes plus tard.

Afin de minimiser la quantité de données écrites par version, celles-ci sont réalisées de manière incrémentale.
Chaque nouvelle version est donc en quelque sorte une différence par rapport aux précédentes.
Ce stockage incrémental est obtenu grâce à une utilisation conjointe de la déduplication et de l'encodage delta.
De plus, comme aucune donnée ne peut être supprimée, nous en profitons pour réaliser un système versionné, qui nous laisse la possibilité d'accéder aux précédentes sauvegardes.


\chapter{Présentation de DNA-Backup}

DNA-Backup, notre proposition, ressemble donc plus à un système de sauvegardes qu'à un système de fichiers.
Chaque nouvelle version vient s'ajouter à la précédente, car il est impossible de supprimer ou de modifier des données
et on applique un \emph{pipeline} de compression sur les données de chaque version afin d'en minimiser la taille.

\section{Pipeline de compression}

Le pipeline est inspiré de celui de Philip Shilane \etal
dans leur travail sur la réplication de sauvegardes à travers un lien de faible bande passante~\cite{shilane2012wan}.
Il est composé d'un étage de déduplication,suivi d'une étape d'encodage delta
et enfin d'un dernier étage de compression.
Ces trois techniques sont basées sur la découverte de similitudes entre différentes zones du flux de données.
La déduplication permet de ne pas réécrire plusieurs fois le même bloc de données si ce bloc existe déjà.
L'encodage delta permet de ne pas avoir à réécrire entièrement un bloc similaire à un bloc existant,
en n'en écrivant que la différence.
La compression est également basée sur ces deux principes, mais elle est appliquée à une plus petite échelle.
Dans DNA-backup, la compression et l'encodage delta sont tous les deux appliqués sur des blocs d'une même taille fixe appelés \emph{chunks}.

Afin d'appliquer de manière efficace les étapes de déduplication et d'encodage delta de notre pipeline,
il nous faut être capable de rapidement retrouver des chunk identiques ou similaires.
Nous utilisons pour cela deux fonctions de hachage qui fournissent des signatures aux propriétés particulières.
L'une de ces signatures, appelée ici \emph{fingerprint} permet d'identifier des chunks identiques à dédupliquer,
l'autre, appelée ici \emph{sketch} permet de trouver des chunks fortement similaires,
lesquels feraient de bons candidats à l'encodage delta.
De cette manière, pour appliquer le pipeline sur une nouvelle version,
il nous suffit d'avoir ces deux informations en mémoire pour chaque chunk déjà écrit.
Seule l'étape d'encodage delta nous forcera à aller chercher le contenu réel d'un chunk similaire,
lorsqu'il aura été trouvé à l'aide de son sketch.

\subsection{Fingerprint}

La \emph{fingerprint} permet d'identifier un bloc de manière unique en fonction de son contenu.
C'est ainsi que l'on peut savoir si deux chunks sont identiques et devraient donc être dédupliqués.
Pour cette application, n'importe quelle fonction de hachage uniformément répartie pourrait convenir.
Il faut simplement calibrer la taille de la signature en fonction du nombre de chunks que le support peut contenir afin d'éviter les collisions.

Dans notre cas précis, nous avons tout de même besoin d'une fonction capable d'être d'appliquée sur une fenêtre glissante, appelée \emph{rolling hash}.
En effet, lorsqu'une nouvelle version est traitée par le pipeline,
il faut appliquer la fonction de hachage sur tous les chunks possibles, octet par octet, des données de cette version.
Une fonction optimisée pour ce type de traitement apportera donc un gain de performances non négligeable.

Nous avons choisi pour cet usage l'empreinte de Rabin~\cite{rabin1981fingerprinting}, car c'est une fonction de hachage sur fenêtre glissante efficiente et populaire et parce qu'elle sera également utilisée pour calculer le \emph{sketch} des chunks.

\subsection{Sketch}

Le \emph{sketch} est ce qu'on appelle un \emph{hash de ressemblance}.
Il correspond en fait à une fonction de hachage dont la répartition n'est pas du tout uniforme,
mais au contraire, fait correspondre la même signature à des données différentes si elles sont similaires.

De manière intuitive, les sketches de similitude fonctionnent en identifiant des portions d'un chunk
qui ne changeraient probablement pas si de faibles variations sont introduites ;
on appelle ces portions des ``features''.
Une des manières de procéder est d'utiliser une fonction de hachage sur une fenêtre glissante de petite taille
(e.g. 32~octets) et de choisir la valeur maximale obtenue en tant que feature.
En utilisant différentes fonctions de hash on peut ainsi obtenir plusieurs features.

La méthode exacte utilisée est celle décrite par Philip Shilane \etal~\cite{shilane2012wan}.
Elle s'appuie sur l'empreinte de Rabin~\cite{rabin1981fingerprinting}
que l'on décline en différentes fonctions de hachage pour obtenir plusieurs features.
Les features ainsi obtenues sont ensuite regroupées pour former un plus petit nombre de ``super-features''.
La valeur d'une super-feature correspond au hash des features sous-jacentes,
donc si deux chunks ont une super-feature en commun,
alors toutes les features correspondantes sont également identiques.
Regrouper les features de cette manière aide à réduire les faux-positifs
et augmente le taux de similarité lorsqu'une correspondance est trouvée.

Augmenter le nombre de features par super-feature augmente la qualité des correspondances,
mais diminue aussi leur nombre.
Augmenter les nombre de super-feature par sketch augmente le nombre de correspondances,
mais nécessite plus d'espace mémoire.
Nous utilisons pour le moment les valeurs choisies par Philip Shilane \etal
au cours de leurs expérimentations~\cite{shilane2012wan},
soit 3 super-features par sketch et 4 features par super-features.
Ces valeurs pourraient être ajustées une fois que de plus amples expériences
sur des jeux de données plus variés auront été réalisés.

\subsection{Index des signatures}
Pour connaitre de manière efficace l'existence d'une fingerprint
ou d'un sketch dans les données existantes du DNA-Drive, nous avons besoin de stocker leur valeur.
En effet, autrement il faudrait les recalculer depuis les données, ce qui serait coûteux.
Ces valeurs sont donc stockées dans deux index.
L'un faisant l'association entre des fingerprints et leur chunk,
l'autre entre des super-features et leurs chunks.
Philip Shilane \etal travaillaient sur de très gros jeux de données
et ne pouvaient pas se permettre de garder l'ensemble de ces index en mémoire,
ils ont donc opté pour n'en garder qu'une partie à l'aide d'un cache~\cite{shilane2012wan}.
Dans notre cas, la quantité de données est suffisamment faible pour qu'on puisse se permettre
de garder la totalité des index en mémoire.
De cette manière nous sommes certains de ne manquer aucune correspondance,
ce qui maximise la qualité de déduplication et l'encodage Delta.

Dans l'hypothèse où l'espace de stockage du DNA-Drive deviendrait beaucoup plus grand,
il restera toujours la possibilité de ne garder qu'un cache en mémoire
et d'utiliser un filtre de Bloom~\cite{bloom1970space} devant le reste de l'index qui serait stocké sur disque.
En effet, même si le temps de recherche dans l'index s'en trouvera augmenté,
il restera très faible par rapport au temps de synthèse.

Cependant, les index de signatures seuls ne sont pas suffisants pour appliquer le pipeline.
Nous avons également besoin du contenu des chunks lors de l'encodage delta.
Or, les temps de lecture rédhibitoires du DNA-Drive nous empêchent de lire à la demande le contenu d'un chunk lorsque l'on en a besoin.
C'est pour cela que nous avons finalement décidé de conserver sur un support de stockage classique une copie des informations stockées sur le DNA-Drive ;
la capacité du DNA-Drive étant pour le moment bien inférieure à ce qui existe sur d'autres supports.


\section{Fonctionnement général}

Le système part donc du principe qu'on dispose, sur un support de stockage classique, d'une copie des données stockées en \ac{adn} appelée le \emph{repo} (Figure~\ref{fig:big-picture}).
Sa raison d'être est d'accélérer la création et la restauration d'une version
en fournissant un accès rapide aux données que contient le DNA-Drive.

\begin{figure*}[ht]
\centering

\begin{tikzpicture}

\draw (0,0) node[anchor=south west] {Ordinateur} rectangle (8, 3.5);

\draw (.5,1) rectangle (3,3) node[midway] {Source};
\draw (5,1) rectangle (7.5,3) node[midway] {Repo};
\draw (10,1) rectangle (12.5,3) node[midway] {DNA-Drive};

\draw[Arrow] (3,2.3) -- (5,2.3) node[midway,above] {Commit};
\draw[Arrow] (7.5,2.3) -- (10,2.3) node[midway,above] {Export};
\draw[Arrow] (5,1.7) -- (3,1.7) node[midway,below] {Restore};
\draw[Arrow] (10,1.7) -- (7.5,1.7) node[midway,below] {Import};

\end{tikzpicture}

\caption{Le repo est une zone intermédiaire entre le dossier source à sauvegarder et le DNA-Drive.}
\label{fig:big-picture}
\end{figure*}

\subsection{Principe de base}

Pour créer une nouvelle version ou en restaurer une existante, DNA-Backup n'utilise donc que le repo.
Il est ensuite possible d'exporter les données du repo vers le DNA-Drive,
ou bien de reconstruire le repo en important les données du DNA-Drive,
par exemple dans le cas d'une défaillance ou bien d'une duplication.

DNA-Backup stocke les données d'une version d'une manière assez particulière.
Chaque version est en fait un \emph{disque virtuel} contenant les données des fichiers mis bout-à-bout.
Ce disque virtuel ne contient aucune métadonnée, seulement le contenu des fichiers
et c'est sur lui qu'on applique le pipeline de compression.
Il n'est donc pas stocké sous la forme d'un segment continu, mais en tant qu'un ensemble de chunks.
La liste des chunks qui le compose fait partie des métadonnées que DNA-Backup doit enregistrer.
C'est à partir de cette liste et du contenu des chunks que l'on peut recréer le disque virtuel.
Mais le disque virtuel à lui seul ne suffit pas à restaurer une version.
En effet, il ne contient aucune métadonnée,
il est donc impossible de savoir à quel fichier correspond une donnée.
DNA-Backup doit donc également sauvegarder pour chaque version une liste de noms de fichiers,
permettant de retrouver dans son disque virtuel où commence et où s'arrête chaque fichier.

Pour restaurer une version, il faut donc dans un premier temps
reconstruire en mémoire le disque virtuel à partir de sa liste de chunks,
puis le découper en fichiers à partir de la liste des noms de fichiers.


\subsection{Contenu du repo}

Le repo est donc une zone intermédiaire qui devra contenir l'ensemble des données écrites sur le DNA-Drive.
Mais contrairement à ce dernier, il est stocké sur un support classique.
Il n'est donc pas nécessaire de l'optimiser autant en terme d'espace de stockage.
D'autant plus qu'il faut également que les données soient facilement accessibles
pour augmenter l'efficacité des algorithmes du commit et du restore.
Toutefois, afin de simplifier les imports et les exports avec le DNA-Drive,
la plupart des données sont stockées de manière quasiment identique.
Il ne faut pas non-plus que l'espace qu'il occupe soit trop important,
la totalité de ses fichiers sont donc compressés.

Le repo tire parti du système de fichier sur lequel il est stocké.
Il organise ses données dans différents fichiers,
chacun rangé dans le dossier de la version à laquelle il appartient.
La Figure~\ref{fig:repo-dir-tree} montre un exemple d'arborescence de repo comportant deux versions.
Un dossier de version est nommé par son numéro et contient trois fichiers de métadonnées :
\verb|files|, \verb|hashes|, et \verb|recipe|, ainsi qu'un dossier \verb|chunks|.

\begin{figure*}[ht]
\centering

\begin{subfigure}[c]{.34\textwidth}
  % \centering % centering ne fonctionne pas du tout avec le dirtree
  \dirtree{%
  .1 repo/.
  .2 00000/.
  .3 chunks/.
  .4 000000000000000.
  .4 000000000000001.
  .4 000000000000002.
  .4 000000000000003.
  .3 files.
  .3 hashes.
  .3 recipe.
  .2 00001/.
  .3 chunks/.
  .4 000000000000000.
  .4 000000000000001.
  .3 files.
  .3 hashes.
  .3 recipe.
  }
  \caption{Exemple d'arborescence d'un repo comportant deux versions.}
  \label{fig:repo-dir-tree}
\end{subfigure}
\hfill
\begin{subtable}[c]{.55\textwidth}
  \centering
  \begin{tabular}{l r r}
  \verb|repo/00000/recipe| &   5076011 &   1.2\% \\
  \verb|repo/00000/files| &      24664 &   0.1\% \\
  \verb|repo/00000/hashes| &   3923672 &   0.9\% \\
  \verb|repo/00000/chunks| & 412263137 &  97.8\% \\
  \end{tabular}
  \caption{
    Répartition des données d'un repo comportant une seule version pour une taille totale de 401 Mio.
    Ce repo a été obtenu à partir d'un dossier de code source,
    il s'agit donc d'un grand nombre de fichiers texte de petite taille, fortement déduplicables et compressibles.
  }
  \label{tab:repo-data-distribution}
\end{subtable}

\caption{Organisation du \emph{repo}.}
\label{fig:repo-organisation}
\end{figure*}

\paragraph{Chunks}
Le dossier \verb|chunks| contient l'ensemble des chunks ajoutés dans cette version.
Chaque chunk est stocké dans un fichier séparé et compressé indépendamment des autres.
De cette manière, il est très facile d'accéder au contenu d'un chunk précis.

\paragraph{Files}
Le fichier \verb|files| contient la liste des fichiers ainsi que leur taille.
C'est grâce à ce fichier qu'on est capable de retrouver
à quel fichier appartient une donnée du disque virtuel d'une version.
Il ne contient en réalité que la différence apportée à cette liste dans cette version
par rapport à la précédente.
Pour reconstruire la véritable liste de fichiers,
il faut donc relire l'ensemble des fichiers \verb|files|
et appliquer les différences les unes à la suite des autres.
De plus, le contenu de ce fichier est finalement compressé avant d'être écrit sur le disque,
afin d'encore économiser de l'espace.

\paragraph{Recipe}
C'est dans le fichier \verb|recipe| que le repo stocke la liste des chunks
permettant de reconstruire le disque virtuel d'une version.
Exactement comme pour le fichier \verb|files|, il est enregistré de manière incrémentale,
en différence par rapport à la version précédente.
Le même processus de reconstruction d'y applique donc.

\paragraph{Hashes}
Pour ne pas avoir à recalculer les signatures (fingerprints et sketches) de tous les chunks du repo lors d'un commit,
on stocke ses valeurs dans un fichier \verb|hashes| séparé.
Celui-ci ne sera pas synthétisé en \ac{adn},
car les données qu'il contient peuvent être recalculées à partir du contenu des chunks.


\subsection{L'export vers le DNA-Drive}

Au lieu de voir le DNA-Drive comme une grille,
on l'imagine comme une liste de \emph{pools} à une seule dimension (Figure~\ref{fig:data-layout}).
De cette manière on peut répartir l'espace de stockage
entre les deux principaux segments de données de DNA-Backup,
les chunks et les métadonnées (recipe + files).
Comme on peut le remarquer sur la Figure~\ref{tab:repo-data-distribution},
les métadonnées ont une taille bien inférieure à celle de l'ensemble des chunks,
en particulier lorsque les fichiers sont grands.
Cependant, il n'est pas possible de savoir à l'avance exactement
quelle proportion sera utilisée par les métadonnées,
car leur taille peut varier en fonction d'un certain nombre de paramètres
(type de fichiers sauvegardés, fréquence de sauvegarde, quantité et taille des fichiers, etc~\textellipsis).
Pour garantir une utilisation maximale de l'espace disponible,
on fait démarrer ces deux segments chacun à une extrémité du DNA-Drive
et on les laisse grandir l'un vers l'autre,
à la manière de la \emph{pile} et du \emph{tas} d'un processus.

\begin{figure}[ht]
\centering

\begin{tikzpicture}[
  start chain = going right,
  node distance = 0,
  Box/.style={draw, minimum width=2em, minimum height=2em, outer sep=0, on chain},
  Brace/.style={decorate,decoration={brace, amplitude=1em, raise=.5em, mirror}}
]
\node[Box] (p0) {$0$};
\node[Box] (p1) {$1$};
\node[Box] (p2) {$2$};
\node[Box] (p3) {$3$};
\node[Box] (p4) {$4$};
\node[Box,minimum width=6em] (ellipsis) {$\cdots$};
\node[Box] (p93) {$93$};
\node[Box] (p94) {$94$};
\node[Box] (p95) {$95$};

\draw[Arrow] (p4.east) to +(2em,0);
\draw[Arrow] (p93.west) to +(-2em,0);

\node (ver) at (0,-3.2em) {version};
\draw[->] (p0.south) to (ver);
\draw[Brace] (p1.south west) to node[black,midway,below=1.5em] {chunks} (p4.south east);
\draw[Brace] (p93.south west) to node[black,midway,below=1.5em,align=center] {metadata\\(recipe+files)} (p95.south east);

\end{tikzpicture}

\caption{Disposition des données.}
\label{fig:data-layout}
\end{figure}

Le tout premier pool du DNA-Dire (numéro 0) est réservé aux \emph{header} des versions
et aux métadonnées globales du repo.

Pour chaque version, il faut conserver la taille que chacun des trois segments
(\verb|chunks|, \verb|recipe| et \verb|files|).
De la même manière qu'il serait impossible de savoir
à quel fichier appartient une donnée sans la liste de fichiers,
il serait impossible sans ces informations, de savoir
à quelle version et à quel segment de données appartient une track.

En ce qui concerne les métadonnées du repo,
un \emph{superblock} serait ajouté dans la toute première track du premier pool.
Celui-ci contiendrait les valeurs des paramètres de DNA-Backup,
à savoir principalement la taille des chunks, mais aussi les paramètres de la fonction de sketch,
l'algorithme utilisé pour la compression utilisé, celui pour les deltas, etc~\textellipsis

Le header d'une version ne comptant que trois valeurs numériques, il ne remplira jamais une track.
Pour éviter de gâcher l'espace restant des tracks de version,
DNA-Backup le remplit avec le contenu des segments de métadonnées (\verb|recipe| + \verb|files|)
jusqu'à ce qu'il soit plein ou que toutes les métadonnées aient été écrites.
S'il reste encore des métadonnées, elles sont écrites à l'endroit prévu initialement.

L'export est actuellement réalisé dans un dossier, dans lequel 96~fichiers
représentant chacun un pool sont créés et remplis avec les données du repo.
Les écritures sont alignées sur la taille du track, laquelle est configurable.
Le repo n'a pas réellement de connaissances de ces valeurs en dehors de l'export.
DNA-Backup est donc en grande partie indépendant du DNA-Drive
et pourrait être utilisé avec d'autres supports de stockages,
ou même en utilisant uniquement le repo,
à la manière d'un logiciel de sauvegardes incrémentales classique.


\subsection{Restaurer depuis le DNA-Drive}

\subsubsection{Reconstruction complète du repo}

Il est possible de reconstruire le \emph{repo} en entier en lisant la
totalité du \emph{DNA-Drive}.

\subsubsection{Restauration de la dernière version}

Il est possible de ne restaurer que la dernière version en lisant dans
un premier temps le \emph{pool} de versions et les quelques \emph{pools}
de métadonnées (environ 2\% de la totalité des données écrites), puis en
lisant tous les \emph{pools} contenant des \emph{chunks} référencés par
la \emph{recipe} de cette version.

\subsubsection{Restauration d'un seul fichier}

Il pourrait être possible (pas pour le moment) de ne restaurer qu'un
seul fichier d'une version en ayant moins de données à lire que pour
restaurer la version complète.

Pour cela, il faudrait en plus stocker en \ac{adn} un mapping \emph{chunk}
décompressé → \emph{pool} contenant ce \emph{chunk} et ainsi n'avoir à
lire que les \emph{pools} contenant des \emph{chunks} de ce fichier.



\chapter{Détails d'implémentation}

% \section{Détails à propos du repo}

% Le fichier *files* correspond au métadonnées de l'arborescence du système de fichier.
% C'est là que sont stockés les métadonnées des fichiers (comme par exemple leur nom et leur taille)
% et c'est grâce à ce segment de données qu'on peut retrouver la position d'un fichier dans le fameux *disque virtuel*

% Conctrètement c'est simplement une liste de structures File :

% \begin{lstlisting}[language=Go]
% type File struct {
%     Path string
%     Size int64
%     Link string
% }
% \end{lstlisting}


% Le fichier *recipe* est celui qui permet de reconstruire le *disque virtuel*. Concrètement on peut le voir comme une liste de *chunks*.
% Sachant que j'ai cette structure d'identifiant pour un chunk:


% \begin{lstlisting}[language=Go]
% type ChunkId struct {
%     Version int
%     Index   uint64
% }
% \end{lstlisting}

% Et 3 types de chunks différents :


% \begin{lstlisting}[language=Go]
% type StoredChunk struct {
%     Id      ChunkId
% }

% type DeltaChunk struct {
%     Source  ChunkId
%     Patch   []byte
%     Size    int
% }

% type PartialChunk struct {
%     Value []byte
% }
% \end{lstlisting}

% /Le PartialChunk s'appelle pour le moment TempChunk dans mon code mais il faut que je change ça./

% Donc la *recipe* contient bien les les patchs ( delta d'un chunk par rapport à un autre) des DeltaChunks comme dit plus haut, mais aussi potentiellement des chunks d'une taille inférieure à 8Kio (typiquement le tout dernier chunk du disque virtuel, mais il peut y en avoir d'autres).

% Il ne faut pas oublier que ces 2 informations sont stockées sous la forme de différences par rapport à la version précédents (ajouts et suppression de fichers et ajouts ou suppression de chunks). 

\section{Algorithme du commit}

\begin{enumerate}
\item
  Chargement des métadonnées du \emph{repo} afin de reconstruire en
  mémoire l'état de la dernière version :

  \begin{itemize}
  \item
    Reconstruction de la \emph{recipe} à partir des deltas de chaque
    version.
  \item
    Reconstruction du listage des fichiers à partir des deltas de chaque
    version (fichier \emph{files}).
  \item
    Reconstruction en mémoire des \emph{maps} de \emph{fingerprints} et
    de \emph{sketches} à partir des fichiers \emph{hashes} de chaque
    version.
  \end{itemize}
\item
  Listage des fichiers de la \emph{source}.
\item
  Concaténation de l'ensemble des fichiers de la source en un disque
  virtuel continu.
\item
  Lecture du \emph{stream} de ce disque virtuel et découpage en
  \emph{chunk} (de 8 Kio actuellement).
\item
  Pour chaque \emph{chunk} du \emph{stream} :

  \begin{enumerate}
  \item
    Calculer sa \emph{fingerprint} (hash classique), si elle est
    présente dans la \emph{map} : le stocker de manière dé-dupliquée
    (sous la forme d'identifiant faisant référence au \emph{chunk}
    trouvé dans la map).
  \item
    Sinon, calculer son \emph{sketch} (hash de ressemblance), s'il est
    présent dans la \emph{map}, le stocker sous la forme de delta
    (calcul de sa différence par rapport au \emph{chunk} trouvé dans la
    map).
  \item
    Sinon, le stocker sous la forme de nouveau bloc (ajout de sa
    \emph{fingerprint} et de son \emph{sketch} dans les \emph{maps} et
    stockage du contenu complet dans un nouveau \emph{chunk}).
  \end{enumerate}
\item
  Calcul des différences entre la nouvelle version et la précédente pour
  les métadonnées (\emph{files} et \emph{recipe}) et stockage des deltas
  ainsi obtenus.
\end{enumerate}

\section{Algorithme du restore}

\begin{enumerate}
\item
  Chargement des métadonnées du \emph{repo} afin de reconstruire en
  mémoire l'état de la dernière version :

  \begin{itemize}
  \item
    Reconstruction de la \emph{recipe} à partir des deltas de chaque
    version.
  \item
    Reconstruction du listage des fichiers à partir des deltas de chaque
    version.
  \end{itemize}
\item
  À partir de la \emph{recipe}, reconstruire le disque virtuel (sous la
  forme d'un \emph{stream}).
\item
  Découper ce \emph{stream} en fonction du listage des fichiers
  (\emph{files}) et réécrire les données dans les fichiers
  correspondants dans le répertoire \emph{destination}.
\end{enumerate}



\chapter{Évaluation de performances}

Le dossier \verb|exp| contient les scripts permettant de reproduire
les expériences. Les scripts ne sont prévus pour fonctionner que sur
Linux.

On utilise le dépôt Git du noyau Linux comme base de donnée de test. Il
s'agit en effet d'une bonne simulation de modification de dossiers, car
l'historique contient toutes les modifications qui ont été apportées
petit à petit à l'ensemble des fichiers.

\section{Bases de comparaison}

Pour évaluer les performances du système DNA-Backup, quatre autres
systèmes de stockage versionnés ont été choisis comme base de
comparaison :

\begin{itemize}
\item
  \textbf{Git diffs}
\item
  \textbf{Git objects}
\item
  \textbf{Tar.gz}
\item
  \textbf{Taille réelle}
\end{itemize}

\subsection{Git diffs}

Ce système utilise le delta généré par la commande \verb|git diff|
pour sauvegarder une nouvelle version. Les données à stocker consistent
donc en une somme de deltas. Pour restaurer les données, il faut
appliquer séquentiellement l'ensemble des deltas jusqu'à obtenir l'état
de la version voulue.

\subsection{Git objects}

Ce système nous permet de simuler un système de fichiers qui ne serait
pas autorisé à modifier des données sur le support tout en gardant la
possibilité de modifier les données. Il s'agit de la manière dont Git
sauvegarde les données des fichiers d'un dépôt. Le contenu de chaque
fichier et de chaque dossier est hashé afin d'en obtenir une signature.
Il est ensuite compressé et stocké sous la forme d'\emph{object}
immuable, référencé par la signature obtenue. Si un fichier est modifié,
il produira une signature différente et sera donc stocké sous la forme
d'un nouvel \emph{object}. Par contre, si deux fichiers ont un contenu
strictement identique, ils produiront alors la même signature et seront
donc automatiquement dé-dupliqués. Les dossiers sont également stockés
en tant qu'\emph{objects}, mais les fichiers qu'ils contiennent sont
référencés non pas par leur nom, mais par leur signature. La
modification d'un fichier entrainera donc l'ajout de nouveaux
\emph{objects} pour l'ensemble des dossiers de la branche contenant ce
fichier. C'est de cette manière que Git est capable de créer un système
de fichiers modifiable à partir d'objets immuables.

\subsection{Tar.gz}

Une technique d'archivage assez classique à laquelle il peut être
intéressant de nous comparer est de stocker chaque version en tant
qu'une nouvelle archive Tar elle-même compressée à l'aide de Gzip. Cette
technique produit des archives d'une taille très réduite, car la
compression est appliquée à l'ensemble des fichiers d'un seul coup,
contrairement à une compression fichier par fichier.

Elle a cependant l'inconvénient de ne pas faire de dé-duplication ni
d'encodage delta, et ne tire donc pas du tout parti des données déjà
écrites sur le support.

\subsection{Taille réelle}

Cette base de comparaison n'est en réalité pas un système viable. Elle
correspond à la taille que prend en réalité le dossier \emph{source} au
moment de la sauvegarde. C'est un indicateur qui permet de se rendre
compte du poids que prendrait la sauvegarde de multiples versions sans
aucune déduplication ou compression.

\subsection{Tableau récapitulatif}

\begin{table*}[ht]

\begin{tabularx}{\textwidth}{L|L|L|L|L|L}

\textbf{Feature} &
\textbf{DNA-Backup} &
\textbf{Git diffs} &
\textbf{Git objects} &
\textbf{Tar.gz} &
\textbf{Taille réelle} \\
\hline

\multirow{2}{=}{Dé\-du\-pli\-ca\-tion} &
Niveau chunk &
\multirow{2}{=}{N/A} &
Niveau fichier &
\multirow{2}{=}{N/A} &
\multirow{2}{=}{N/A} \\ \cline{2-2} \cline{4-4}
& Transversal aux versions & & Transversal aux versions & \\
\hline

\multirow{2}{=}{Delta-encoding} &
Niveau chunk &
Niveau version &
\multirow{2}{=}{N/A} &
\multirow{2}{=}{N/A} &
\multirow{2}{=}{N/A} \\ \cline{2-3}
& Transversal aux versions & Par rapport à la précédente & & \\
\hline

Com\-pres\-sion &
Niveau chunk &
Niveau version &
Niveau fichier &
Niveau version &
N/A \\
\hline

Res\-tau\-ra\-tion de la dernière version &
Lecture des métadonnées puis des chunks de cette version (répartis dans différents pools) &
Lecture de la totalité du DNA-Drive &
Lecture récursive des différents objets composant la version (répartis dans différents pools) &
Lecture de la zone correspondant à la dernière version &
Lecture de la zone correspondant à la dernière version \\

\end{tabularx}

\caption{Tableau récapitulatif}
\label{tab:recap-table}
\end{table*}

\section{Nombre d'octets par version}

\subsection{Légende}

\begin{itemize}
\item
  \verb|4k_export| : le système DNA-Backup avec des blocs de 4 Kio.
\item
  \verb|8k_export| : le système DNA-Backup avec des blocs de 8 Kio.
\item
  \verb|diffs| : une somme de diffs Git minimales Gzippées.
\item
  \verb|nopack| : le dossier `objects de Git, contenant l'ensemble des
  données des fichiers et dossiers d'un dépôt.
\item
  \verb|targz| : une somme d'archives Tar Gzippées.
\item
  \verb|real| : le poids réel de chaque version et donc l'espace
  nécessaire à stocker l'ensemble des versions de manière
  non-dédupliquées.
\end{itemize}

\subsection{Résultats}


% TODO: use real data
\begin{table*}[ht]
\centering
\begin{tabularx}{\textwidth}{@{}RRRRRR}
\textbf{DNA 4k} &
\textbf{DNA 8k} &
\textbf{Git diffs} &
\textbf{Git objects} &
\textbf{Tar.gz} &
\textbf{Taille réelle} \\
\hline
46 080 540 & 46 021 380 & 47 011 621 & 63 237 214 & 47 582 831 & 201 468 483 \\
8 160 & 13 260 & 2 625 & 88 673 & 47 581 355 & 201 464 200 \\
6 453 540 & 8 091 660 & 3 270 798 & 26 668 800 & 48 698 909 & 205 998 038 \\
205 020 & 108 120 & 496 & 24 840 & 48 699 087 & 205 998 035 \\
214 200 & 121 380 & 1 475 & 318 054 & 48 699 763 & 205 998 379 \\
255 000 & 162 180 & 3 271 631 & 107 650 & 47 591 733 & 201 464 426 \\
393 720 & 358 020 & 99 337 & 2 758 950 & 47 582 211 & 201 412 617 \\
67 320 & 78 540 & 127 793 & 561 940 & 47 578 053 & 201 407 077 \\
155 040 & 75 480 & 19 221 & 10 035 & 47 594 809 & 201 459 655 \\
286 620 & 205 020 & 250 581 & 1 203 017 & 47 721 093 & 202 024 780 \\
39 780 & 38 760 & 19 555 & 550 478 & 47 726 206 & 202 033 937 \\
159 120 & 80 580 & 203 & 45 564 & 47 726 209 & 202 033 922 \\
182 580 & 115 260 & 12 419 & 284 765 & 47 727 829 & 202 049 528 \\
13 260 & 14 280 & 5 823 & 76 009 & 47 731 771 & 202 057 683 \\
23 460 & 28 560 & 13 370 & 528 744 & 47 735 296 & 202 070 421 \\
27 540 & 33 660 & 10 837 & 374 954 & 47 736 886 & 202 070 084 \\
68 340 & 81 600 & 69 707 & 498 919 & 47 770 410 & 202 207 821 \\
\hline
54 633 240 & 55 627 740 & 54 187 492 & 97 338 606 & 813 484 451 & 3 443 219 086 \\

\end{tabularx}
\caption{Commits journaliers}
\label{tab:commits-daily}
\end{table*}


% TODO: use real data
\begin{table*}[ht]
\begin{tabularx}{\textwidth}{@{}RRRRRR}
\textbf{DNA 4k} &
\textbf{DNA 8k} &
\textbf{Git diffs} &
\textbf{Git objects} &
\textbf{Tar.gz} &
\textbf{Taille réelle} \\
\hline
46 086 660 & 46 003 020 & 47 003 541 & 63 221 561 & 47 569 854 & 201 412 617 \\
701 760 & 820 080 & 395 080 & 6 358 049 & 47 719 963 & 202 057 683 \\
6 293 400 & 7 983 540 & 2 994 599 & 25 581 927 & 48 699 737 & 205 995 565 \\
206 040 & 109 140 & 407 & 50 816 & 48 699 759 & 205 995 603 \\
225 420 & 142 800 & 8 679 & 401 380 & 48 699 627 & 205 997 073 \\
1 299 480 & 1 707 480 & 579 422 & 6 943 226 & 48 734 056 & 206 089 868 \\
952 680 & 1 248 480 & 360 710 & 4 799 959 & 48 837 441 & 206 640 167 \\
1 425 960 & 1 831 920 & 738 359 & 4 983 830 & 48 887 777 & 206 826 648 \\
1 770 720 & 2 091 000 & 1 389 502 & 7 767 438 & 49 298 535 & 209 320 664 \\
479 400 & 727 260 & 146 129 & 2 899 284 & 49 332 106 & 209 471 170 \\
168 300 & 235 620 & 47 436 & 1 385 569 & 49 336 133 & 209 495 372 \\
134 640 & 236 640 & 37 183 & 1 808 602 & 49 337 222 & 209 501 585 \\
90 780 & 122 400 & 23 924 & 1 555 868 & 49 337 681 & 209 507 160 \\
3 088 560 & 3 953 520 & 1 404 256 & 11 037 483 & 49 932 666 & 211 870 188 \\
4 987 800 & 6 165 900 & 2 326 692 & 17 577 030 & 50 210 539 & 212 932 833 \\
993 480 & 1 378 020 & 304 617 & 6 594 518 & 50 297 457 & 213 246 213 \\
684 420 & 900 660 & 258 512 & 4 016 392 & 50 398 693 & 213 642 553 \\
\hline
69 589 500 & 75 657 480 & 58 019 048 & 166 982 932 & 835 329 246 & 3 540 002 962 \\

\end{tabularx}
\caption{Commits hebdomadaires}
\label{tab:commits-weekly}
\end{table*}


% TODO: use real data
\begin{table*}[ht]
\begin{tabularx}{\textwidth}{@{}RRRRRR}
\textbf{DNA 4k} &
\textbf{DNA 8k} &
\textbf{Git diffs} &
\textbf{Git objects} &
\textbf{Tar.gz} &
\textbf{Taille réelle} \\
\hline
46 080 540 & 46 021 380 & 47 011 621 & 63 237 214 & 47 582 831 & 201 468 483 \\
8 160 & 13 260 & 2 625 & 88 673 & 47 581 355 & 201 464 200 \\
6 453 540 & 8 091 660 & 3 270 798 & 26 668 800 & 48 698 909 & 205 998 038 \\
205 020 & 108 120 & 496 & 24 840 & 48 699 087 & 205 998 035 \\
214 200 & 121 380 & 1 475 & 318 054 & 48 699 763 & 205 998 379 \\
255 000 & 162 180 & 3 271 631 & 107 650 & 47 591 733 & 201 464 426 \\
393 720 & 358 020 & 99 337 & 2 758 950 & 47 582 211 & 201 412 617 \\
67 320 & 78 540 & 127 793 & 561 940 & 47 578 053 & 201 407 077 \\
155 040 & 75 480 & 19 221 & 10 035 & 47 594 809 & 201 459 655 \\
286 620 & 205 020 & 250 581 & 1 203 017 & 47 721 093 & 202 024 780 \\
39 780 & 38 760 & 19 555 & 550 478 & 47 726 206 & 202 033 937 \\
159 120 & 80 580 & 203 & 45 564 & 47 726 209 & 202 033 922 \\
182 580 & 115 260 & 12 419 & 284 765 & 47 727 829 & 202 049 528 \\
13 260 & 14 280 & 5 823 & 76 009 & 47 731 771 & 202 057 683 \\
23 460 & 28 560 & 13 370 & 528 744 & 47 735 296 & 202 070 421 \\
27 540 & 33 660 & 10 837 & 374 954 & 47 736 886 & 202 070 084 \\
68 340 & 81 600 & 69 707 & 498 919 & 47 770 410 & 202 207 821 \\
\hline
54 633 240 & 55 627 740 & 54 187 492 & 97 338 606 & 813 484 451 & 3 443 219 086 \\

\end{tabularx}
\caption{Commits Mensuels}
\label{tab:commits-monthly}
\end{table*}


% Bibliography
\bibliography{doc.bib}

% Annexes
\appendix

% Acronyms
\include{assets/acronyms.tex}

\chapter{Documentation de la CLI}

DNA-Backup est un programme s'utilisant en \ac{cli}.
Trois commandes sont disponibles :

\begin{itemize}
  \item \verb|commit| : pour ajouter une nouvelle version au \emph{repo}.
  \item \verb|restore| : afin de restaurer la dernière version depuis le \emph{repo}
  \item \verb|export| : pour générer un export à partir des données du \emph{repo}
\end{itemize}


\end{document}
